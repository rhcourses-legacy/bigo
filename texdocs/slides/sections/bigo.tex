\begin{frame}
    \begin{block}{Bisher: Informelle Komplexitätsabschätzungen}
        \begin{itemize}
            \item Laufzeitabschätzungen in Abhängigkeit der Größe einer Datenstruktur
            \begin{itemize}
                \item z.B. Länge einer Liste oder Anzahl der Elemente eines Baumes
            \end{itemize}
            \item<2-> Beobachtung: Laufzeit wird i.d.R. \emph{ungenau} angegeben.
            \begin{itemize}
                \item z.B. Schleifendurchläufe zählen, aber nicht die Anzahl der Operationen innerhalb der Schleife
                \item z.B. geschachtelte Schleifen berücksichtigen, hintereinander ausgeführte Schleifen aber nicht
            \end{itemize}
        \end{itemize}
    \end{block}

    \begin{block}<3->{Ziel: Formalisierung dieser Ungenauigkeiten}
        \begin{itemize}
            \item Wie kommen diese Abschätzungen zustande?
            \item Welche Operationen müssen gezählt werden?
        \end{itemize}
    \end{block}
\end{frame}

\begin{frame}
    \todo[inline]{Beispiele für Ungenauigkeiten}
\end{frame}
    
\begin{frame}
    \begin{defblock}{$O$-Notation}
        \todo[inline]{Definition hinzufügen.}
    \end{defblock}
\end{frame}
