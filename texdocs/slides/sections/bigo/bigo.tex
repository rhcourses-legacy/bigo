\begin{frame}
    \begin{block}{Bisher: Informelle Komplexitätsabschätzungen}
        \begin{itemize}
            \item Laufzeitabschätzungen in Abhängigkeit der Größe einer Datenstruktur
            \begin{itemize}
                \item z.B. Länge einer Liste oder Anzahl der Elemente eines Baumes
            \end{itemize}
            \item<2-> Beobachtung: Laufzeit wird i.d.R. \emph{ungenau} angegeben.
            \begin{itemize}
                \item z.B. Schleifendurchläufe zählen, aber nicht die Anzahl der Operationen innerhalb der Schleife
                \item z.B. geschachtelte Schleifen berücksichtigen, hintereinander ausgeführte Schleifen aber nicht
            \end{itemize}
        \end{itemize}
    \end{block}

    \begin{block}<3->{Ziel: Formalisierung dieser Ungenauigkeiten}
        \begin{itemize}
            \item Wie kommen diese Abschätzungen zustande?
            \item Welche Operationen müssen gezählt werden?
        \end{itemize}
    \end{block}
\end{frame}

\begin{frame}
    \begin{exblock}{Maximum einer Liste bestimmen}
        \inputsrcfile[slidelisting, xleftmargin=-3em, basicstyle=\ttfamily\scriptsize]{sections/bigo/code/ListSearch.java}[9-15]
    \end{exblock}
    \begin{block}<2->{Komplexität}
        \begin{itemize}
            \item $n$ Schleifendurchläufe (Aufrufe von \code{Math.max})
            \item Komplexitätsklasse: \olin
        \end{itemize}
    \end{block}
\end{frame}

\begin{frame}
    \begin{exblock}{Differenz zw. Minimum und Maximum bestimmen}
        \inputsrcfile[slidelisting, xleftmargin=-3em, basicstyle=\ttfamily\scriptsize]{sections/bigo/code/ListSearch.java}[20-30]
    \end{exblock}
    \begin{block}<2->{Komplexität}
        \begin{itemize}
            \item $2n$ Aufrufe von \code{Math.max} oder \code{Math.min}
            \item Komplexitätsklasse: \olin
            \begin{itemize}
                \item \alert{Warum nicht \bigo{2n}?}
            \end{itemize}
        \end{itemize}
    \end{block}
\end{frame}

\begin{frame}
    \begin{exblock}{Minimale Differenz von Elementen bestimmen}
        \inputsrcfile[slidelisting, xleftmargin=-3em, basicstyle=\ttfamily\scriptsize]{sections/bigo/code/ListSearch.java}[35-43]
    \end{exblock}
    \begin{block}<2->{Komplexität}
        \begin{itemize}
            \item $n$ Durchläufe der äußeren Schleife
            \item pro Durchlauf: $\leq n$ Durchläufe der inneren Schleife
            \begin{itemize}
                \item \alert{Warum $\leq n$ und nicht genauer?}
            \end{itemize}
            \item Komplexitätsklasse: \osquare
        \end{itemize}
    \end{block}
\end{frame}

\begin{frame}
    \begin{block}{Beobachtungen}
        \begin{itemize}
            \item Komplexitätsklassen geben nur die Größenordnung an.
            \item Konstante Faktoren und nicht-dominante Terme werden vernachlässigt.
        \end{itemize}
    \end{block}
    \begin{block}<2->{Beispiele}
        \vspace{-3ex}
        \begin{align*}
            \olin &= \bigo{2n} = \bigo{\frac{n}{2}} \\
            \osquare &= \bigo{n^2+n+1} = \bigo{\left(\frac{n}{2}\right)^2} \\
            \onlog &= \bigo{2n \log n  + 50n}
        \end{align*}
    \end{block}
\end{frame}

\begin{frame}
    \begin{block}{Intuition:}
        \begin{itemize}
            \item Der Unterschied zwischen \olin und \bigo{2n} kann durch schnellere Hardware ausgeglichen werden.
            \item Ebenso der Unterschied zwischen \osquare und \bigo{2(n^2)}.
            \item Der Unterschied zwischen \olin und \osquare kann nicht so einfach kompensiert werden.
            \item Das Verhalten von Polynomen (Funktionen) wird i.W. vom \emph{Leitterm} bestimmt.
        \end{itemize}
    \end{block}
    \begin{block}<2->{Ziel bei der Entwicklung:}
        \begin{itemize}
            \item Komplexitätsklasse möglichst klein halten.
            \item \alert{Komplexität kann nicht durch Hardware ausgeglichen werden!}
            \item Konstante oder lineare Faktoren sind weniger von Bedeutung.
        \end{itemize}
    \end{block}
\end{frame}
    
\begin{frame}
    \begin{defblock}{$O$-Notation}
        \todo[inline]{Definition hinzufügen.}
    \end{defblock}
\end{frame}
