\begin{frame}
    \begin{block}{Ziel: Prüfung, ob zwei Listen die gleichen Elemente haben}
        \begin{itemize}
            \item Gegeben: Zwei Listen von Zahlen \code{A} und \code{B}
            \item Ergebnis: \code{true}, falls jedes Element aus Liste \code{A} auch in Liste \code{B} vorkommt und umgekehrt.
        \end{itemize}
    \end{block}
    \begin{block}<2->{Naiver Ansatz}
        \begin{itemize}
            \item Durchlaufe Liste \code{A}.
            \begin{itemize}
                \item Suche jedes Element in Liste \code{B}.
            \end{itemize}
            \item Wiederhole für Liste \code{B}.
        \end{itemize}
    \end{block}
    \begin{block}<3->{Komplexität}
        \begin{itemize}
            \item $n$ Durchläufe der äußeren Schleife
            \item pro Durchlauf: $\leq n$ Durchläufe der inneren Schleife
            \item Komplexitätsklasse: \osquare
        \end{itemize}
    \end{block}
\end{frame}

\begin{frame}
    \begin{block}{Ziel: Prüfung, ob zwei Listen die gleichen Elemente haben}
        \begin{itemize}
            \item Gegeben: Zwei Listen von Zahlen \code{A} und \code{B}
            \item Ergebnis: \code{true}, falls jedes Element aus Liste \code{A} auch in Liste \code{B} vorkommt und umgekehrt.
        \end{itemize}
    \end{block}
    \begin{block}<2->{Optimierte Lösung}
        \begin{itemize}
            \item Sortiere beide Listen.
            \item Vergleiche die Listen in einem einzigen Durchlauf.
        \end{itemize}
    \end{block}
    \begin{block}<3->{Komplexität}
        \begin{itemize}
            \item Sortieren: \onlog
            \item Vergleichen: \olin
            \item \alert{Gesamt}: $\onlog + \olin = \alert{\onlog}$
        \end{itemize}
    \end{block}
\end{frame}
