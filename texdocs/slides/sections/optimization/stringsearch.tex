\begin{frame}
    \begin{block}{Ziel: Suche nach einem String in einem Text}
        \begin{itemize}
            \item Gegeben: Ein Text (String) der Länge $n$ und ein zu suchender String der Länge $m$.
            \item Ergebnis: Alle Positionen, an denen der Suchstring vorkommt.
        \end{itemize}
    \end{block}
    \begin{block}<2->{Naiver Ansatz}
        \begin{itemize}
            \item Durchlaufe den gesamten Text.
            \begin{itemize}
                \item An jeder Position vergleiche den dortigen Teilstring mit dem gesuchten String.
            \end{itemize}
        \end{itemize}
    \end{block}
    \begin{block}<3->{Komplexität}
        \begin{itemize}
            \item $n$ Durchläufe der äußeren Schleife
            \item pro Durchlauf: $m$ Schritte für den Vergleich.
            \item Komplexitätsklasse: \bigo{n \cdot m}.
        \end{itemize}
    \end{block}
\end{frame}

\begin{frame}
    \begin{block}{Ziel: Suche nach einem String in einem Text}
        \begin{itemize}
            \item Gegeben: Ein Text (String) der Länge $n$ und ein zu suchender String der Länge $m$.
            \item Ergebnis: Alle Positionen, an denen der Suchstring vorkommt.
        \end{itemize}
    \end{block}
    \begin{block}<2->{Optimierung}
        \begin{itemize}
            \item Durchlaufe den gesamten Text.
            \begin{itemize}
                \item An jeder Position vergleiche den dortigen Teilstring mit dem gesuchten String.
                \item Bei Nicht-Übereinstimmung berechne, wie weit gesprungen werden kann.
            \end{itemize}
        \end{itemize}
    \end{block}
\end{frame}

\begin{frame}
    \begin{block}{Ziel: Suche nach einem String in einem Text}
        \begin{itemize}
            \item Gegeben: Ein Text (String) der Länge $n$ und ein zu suchender String der Länge $m$.
            \item Ergebnis: Alle Positionen, an denen der Suchstring vorkommt.
        \end{itemize}
    \end{block}
    \begin{block}<2->{Optimierung bei häufiger Suche}
        \begin{itemize}
            \item Baue einen \alert{Suchindex} auf:
            \item Z.B. ein Präfixbaum, der für mögliche Suchbegriffe die Positionen im Text enthält.
            \item Kann aus einer Datenbank häufiger Anfragen erstellt werden.
            \item Kann nebenbei erstellt und aktualisiert werden.
        \end{itemize}
    \end{block}
\end{frame}
